\documentclass[10pt,a4paper]{article}
\usepackage[margin=0.5in,footskip=0.25in]{geometry}
\usepackage[utf8]{inputenc}
\usepackage[russian]{babel}
\usepackage[OT1]{fontenc}
\usepackage{amsmath}
\usepackage{amsfonts}
\usepackage{amssymb}
\RequirePackage{ifthen}
\usepackage{parallel}
\usepackage{blindtext}

\newcommand{\lang}[2]{\begin{Parallel}{9cm}{9cm}\ParallelLText{#1}\ParallelRText{#2}\ParallelPar \end{Parallel}}
\newcommand{\ai}[2]{
	\lang{
		\texttt{#1}
	}{
		\texttt{#2}
	}
}

\newcommand{\ainame}{AI NAME GOES HERE}
\newcommand{\mainname}{MAIN CHAR NAME GOES HERE}
\newcommand{\auxname}{SIDEKICK CHAR NAME GOES HERE}
\newcommand{\corpname}{COMPANY NAME GOES HERE}
\newcommand{\ceoname}{CEO NAME GOES HERE}


\newcommand{\sceneplaceholder}[1]{SCENE PLACEHOLDER GOES HERE: #1. \blindtext}
\begin{document}
\lang{Text in Russian goes here}{Text in English will go here}
\lang{Text in Russian goes here}{Text in English shall go here}
\iffalse
Main idea for text: explore what happens if executive meddling is invoked on important technology (like internet is being broken by governments everywhere)

Story outline: 
tech inventor makes tech
some influencer (govt, CEO, ...) tries to manipulate tech in order to serve some (not-public) interest
(inventor must think manipulation is good because of their own story)
influencer causes tech to have modification that serves non-public interest
original creator breaks tech (TODO: add some choice of whether to break; keeping it must have benefit)
people are unhappy when tech is modified, but less so when tech broken
people independetly clone tech, so influencer incapable of controlling it

Specifics:
- tech is voice-activated AI, controls (pre-existing) connected devices 
- influencer is ceo of company, was approached by police to assist in crimebusting
- modification is surveillance and automatic reporting (for reasons of law enforcement)
- reason inventor approves is his home had things stolen from; thinks that if surveillance was in place, it would've been fixed (he already made surveillance for his home)
%- reason it is broken is false triggering (that cannot be fixed: see https://www.theregister.co.uk/2017/11/17/its_artificial_its_intelligent_its_in_my_home_and_its_gone_bonkers/)
- false triggering cannot be fixed because AI does not feel empathy with owners, plus new crimes that exploit the AI system.

\fi

\ai{Инициализация... Система запущена.}{Initializing... System online.}
\lang{-- Так, наконец-то загрузилось. \ainame, сколько будет семь плюс восемь?}{``Oh good, it's finally booted. \ainame, what is seven plus eight?''}
\ai{Семь плюс восемь равно пять тысяч семьдесят четыре точка девять девять девять девять девять де- Выключение.}{Seven plus eight is five thousand seventy four point nine nine nine nine nine nine ni- Shutting down.}
\lang{-- Что на \emph{этот} раз сломалось? -- спросил \mainname, вставая из-за стола.}{``Alright, what's broken \emph{this} time?'' \mainname asked, getting up from his desk.}
\lang{-- Я только что накатил новую версию на бэкенд, тебе стоит скачать новую прошивку, -- послышался голос \auxname из-за серверной стойки.}{``I just pushed a new version out on the backend, you might want to download the new firmware version,'' he heard \auxname say from behind the server rack.}
\lang{-- Ах, так и есть. Сейчас быстренько скачаю, -- сказал \mainname, и все экраны в офисе внезапно показали крутащуюся штуку и слово ``\texttt{ЗАГРУЗКА}''.}{``Ah, so it is. Okay, let me just do that real quick,'' said \mainname, and in a moment, every screen in the office switched over to showing a pretty spinning wheel and the word ``\texttt{DOWNLOADING}''.}
\lang{Обыкновенно, такое обширное прерывание остановило бы всю работу на день, потому что все ушли бы в бар на очень длинный обеденный перерыв. Но \mainname пошел к торговому автомату и, оглянувшись, посчитал, сколько людей пришли сегодня.}{Ordinarily, such a major interruption would halt all work for the day, as everyone in the office rushed to the pub for an extended lunch break. But as \mainname walked over to the vending machine, he looked around the office, trying to count how many people were in today.}
\lang{-- Семь. Две недели подрад уже как.}{``Seven. That's 2 weeks in a row.''}
\lang{Несмотря на то, что \corpname -- самый известный производитель умных усройств для дома, доходы компании оставляли желать лучшего. Продажи падали последние два года, в основном потому что более простые и более дешёвые альтернативы стали значительно доступнее. \ceoname, директор компании, сильно заблуждается, когда говорит, что в каждом доме должен быть автономный дрон для доставки еды.}{Despite \corpname's reputation as the leading provider of connected household technolody, its profit margins left a lot to be desired. Sales have been down for the last 2 years, mostly because cheaper, simpler alternatives to their products have become widely available. Despite what \ceoname, the company's CEO, would have you believe, not everyone wants an autonomous drone for food delivery.}
\lang{-- Так, вы все сегодня хорошо поработали. Всем напитков за мой счёт! -- сказал \mainname, закружая в дрон бутылки с газировкой. Провожая его взглядом, он вспомнил, что в компании TransCorp, конкуренте \corpname, теперь используют военных десантных роботов как меру снижения затрат. Хотя бы тут до этого не дошло, хотя \ceoname и не на такое способен.}{``Okay, good work everybody. Let's have a round of drinks on the house!'' said \mainname, loading up the drone with soda cans. As it flew off over the empty cubicles, he remembered that one of their competitors, TransCorp, has decided to use military assault robots as a cost-cutting measure. At least it didn't come to that here, he thought, although he wouldn't put it past \ceoname.}
\lang{Вот этот новый продукт, \ainame, должен решить все трудности компании. Под брендом \corpname продаются и микроволновки, и окна, и дверные звонки, и строго говоря это умные устройства, но они не ведут себя умно в отсутствии того, к чему подключаться. А если у вас есть \ainame, то вам надо только сказать, что вам нужно, и все устройства \corpname будут для этого задействованы. Вам лишь надо сказать о том, что хотите посмотреть фильм, и сразу попкорн готов, свет притушен, а телевизор включен и фильм скачан.}{Now this new product, \ainame, is supposed to fix all the company's problems. While a \corpname-branded microwave oven, or window, or doorbell, is technically a connected device, when there is nothing to connect to, it just works like a normal appliance. But in the presence of \ainame, you only need to say what you want to be done and the AI will use all your \corpname devices to do so. Meaning you just say that you want to watch a movie, and right away the popcorn is ready, the lights are dimmed, and your TV is ready to play.}
\lang{Так в теории. Над проектом работает критически мало людей, поэтому больше времени все ждут, пока устройство -- а оно само выглядит, как банка газировки -- загрузится, подключится к серверам \auxname, а затем заявит, что семь плюс восемь -- пять тысяч семьдесят четыре точка девять девять девять девять девять. И даже это не было бы слишком плохо, но ведь понятно же, что директор зайдёт в самый неподходящий момент и скажет --}{That's the theory, anyway. With the project being so understaffed, it becomes more about waiting for the device -- itself resembling a soda can -- to boot, connect to \auxname's servers, and then claim that seven plus eight is five thousand seventy four point nine nine nine nine nine nine. That alone would not be so bad, except you just know the CEO's going to come in at the worst possible time and say --}

\sceneplaceholder{Maker sees news report about too much crimes, maker thinks to add helpful surveillance, CEO approves.}

\sceneplaceholder{Maker looks at analytics for sales, CEO tells police for testing are in. Police get criminal to showroom house, criminal fails, police approve.}

\sceneplaceholder{Surveillance implemented mandatorily citywide, but} % everyone is depressed (show this with one person, possibly suicide?) 


\end{document}
